\setlength{\parindent}{4em}
\chapter*{Abstract}


\hspace{4em} People with disorders of consciousness (DOC) are vulnerable to misdiagnosis which can negatively affect their rehabilitation process. The incorrect diagnosis of people with DOC is common, reaching 43 \%. This has possible implications for decisions regarding the provision of health care to this group. Diagnosis of this population depends on the assessment of their behavioral responses to stimuli. The intentionality and types of behavior exhibited by people in a vegetative state (VS) and a minimally conscious state (MCS) can be difficult to distinguish and subtle signs of consciousness can go unnoticed.
It is widely recognized that the use of standardized and sensitive behavioral assessment scales, such as the Sensory Modality Assessment and Rehabilitation Technique (SMART), can help healthcare professionals to identify subtle signs of awareness.
SMART is an assessment tool that combines communication, motor and sensory modalities to diagnose the condition of patients who have suffered severe brain injuries. This method is quite credible and accepted by the healthcare community that deals with this clinical population. It requires and consumes many resources in making the diagnosis.

However, less experienced SMART assessors can be misled by some types of patient responses and even in the analysis of session data, namely in different diagnostic limit zones. Hence a second opinion based on Machine Learning can prove to be very useful. In addition, diagnosis is made session by session and the cumulative diagnostic certainty as the sessions progress.
This tool can be detected as well as the expert assessors, in the future it can be very useful to detect in advance (in a smaller number of sessions) the state of consciousness of the patient to be analyzed.

SMART evaluation has already been explored with statistical software such as statistical package for the social sciences (SPSS) combining analysis methods and techniques such as analysis of variance (ANOVA), etc.
So far, no machine learning methods have been found in partnership with this technique and diagnostic tools (SMART).  

The best diagnosis, through a second opinion performed by the machine, is expected to increase the confidence level in decision-making by SMART assessors. More protected and less subject to criticism of negligence, data that possible errors if detected, can be bridged and safeguarded or at least become noticeable therefore, it leads to higher hit rates. Minimizing the allocation of time to human resources for this specific task, can be beneficial for these professionals due to the useful / effective time to perform tasks (elimination of extra hours to do a task that was previously done).
Institutions: speeds up the prognosis and diagnosis process, making it financially convenient to make professionals more available, resulting in greater performance and efficiency, with the possibility of performing other essential tasks.\\


\vspace*{10mm}\noindent
\textbf{Keywords}: Machine Learning, disorders of consciousness, diagnosis, SMART, minimally conscious state, vegetative state, brain injury,

\chapter*{Resumo}
\hspace{4em} Pessoas com perturbações de consciência (PdC) são vulneráveis a diagnósticos errados que podem afetar negativamente o seu processo de reabilitação. O diagnóstico incorreto de pessoas com PdC é comum, podendo atingir os 43\%. Isto acarreta possíveis implicações nas decisões relacionadas com a prestação de cuidados de saúde desta população. o diagnóstico desta população depende da avaliação das suas respostas comportamentais a estimulação. A intencionalidade e os tipos de comportamentos exibidos por pessoas em estado vegetativo e estado de consciência mínima, podem ser difíceis de distinguir e sinais subtis de consciência podem passar despercebidos. É amplamente reconhecido que o uso de escalas de avaliação comportamental padronizadas e sensíveis, tal como o Sensory Modality Assessment and Rehabilitation Technique (SMART), pode ajudar os profissionais de saúde a identificar sinais subtis de consciência. 
O SMART é um instrumento de avaliação que combina funções comunicacionais, motoras e de aferição de sentidos para diagnosticar o estado de pacientes que sofreram lesões cerebrais graves. Este método é bastante credível e aceite pela comunidade da área da saúde  que lida com esta população clínica. Ele requer e consome muitos recursos na elaboração do diagnóstico.

Contudo, avaliadores SMART menos experientes podem ser induzidos em erro por algum tipos de respostas dos pacientes e mesmo na análise dos dados das sessões nomeadamente em zonas limite de diagnósticos distintos. 
Daí uma segunda opinião com base em Machine Learning pode vir a ser muito útil. Além disso, o diagnóstico é feito sessão a sessão sendo a certeza de diagnóstico cumulativa no progredir das sessões.

Esta ferramenta se detetar tão bem como os avaliadores expert, pode no futuro ser muito útil a detetar antecipadamente (num menor número de sessões) o estado de consciência do paciente analisado.
- A avaliação SMART já foi explorada através de software estatístico como SPSS usando métodos e técnicas de análise ANOVA, etc.
Até ao momento  não foram encontrados  métodos de machine learning em parceria desta técnica e ferramenta de diagnóstico (SMART).  
 
  Melhor diagnóstico, através de uma segunda opinião realizada pela máquina é expectável que aumente o índice de confiança na sua tomada de decisão por parte dos avaliadores SMART . Mais resguardados e menos sujeitos a críticas de negligência, dados que possíveis erros se detectados, possam ser colmatados e salvaguardados ou pelos menos se tornem perceptíveis portanto, conduz a taxas de acerto mais elevadas. Redução dos tempos de alocação a recursos humanos destinados a esta tarefa específica, pode ser benéfico para estes profissionais pelo tempo útil de realização de tarefas (eliminação das horas a mais a fazer uma tarefa já realizada outrora). 
Instituições: agiliza o processo de prognóstico e diagnóstico, financeiramente conveniente tornar profissionais mais libertos consequente maior rendimento e eficácia com possibilidade de realização de outras tarefas primordiais.

\vspace*{10mm}\noindent
\textbf{Keywords}: Aprendizado Máquina, Perturbações de consciência PdC,diagnóstico de consciência, SMART, estado de consciência mínimo, estado vegetativo, lesão cerebral
